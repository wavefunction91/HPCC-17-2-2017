\documentclass{beamer}
\usetheme[compress]{Singapore}
\setbeamerfont{footnote}{size=\tiny}


\usepackage{amsmath}
\usepackage{mathrsfs}
\usepackage{amssymb}
\usepackage{feynmp}
\usepackage{bm}


\newcommand{\bpar}[1]{\left( #1 \right)}                  % big parentheses
\newcommand{\bra}[1]{\left \langle #1 \right \vert}
\newcommand{\ket}[1]{\left \vert #1 \right \rangle}
\newcommand{\comm}[2]{\left[ #1 , #2\right]}
\newcommand{\qop}[1]{\mathscr{#1}}
\newcommand{\mat}[1]{\boldsymbol{#1}}
\newcommand{\dbar}[2]{\left\langle \left. #1 \right\vert \left\vert #2 \right.\right\rangle}

\title[]{Studying Semi--Classical Molecular Light--Matter Interaction through
Time--Dependent Density Functional Theory (TD-DFT)}
\author[Li Research Group, University of Washington]{
\\[1\baselineskip]
David Williams--Young \\
Department of Chemistry \\
University of Washington \\
Li Research Group
}

\date{\today}

\begin{document}

% Title Page
\begin{frame}
\titlepage
\end{frame}


%%%%% OUTLINE SECTION
\section{Outline}


% Outline
\begin{frame}
\frametitle{Outline}

\begin{itemize}
  \item Semi--Classical Molecular Scattering of Polarized Light
  \item Density Functional Theory (DFT)
  \item Real--Time Time--Dependent Density Functional Theory (RT-TD-DFT)
  \item Chebyshev Expansion of the Quantum Propagator
  \item Some Fun Results
\end{itemize}
\end{frame}

%%%%% THEORY SECTION
\section{Theory}

% Molecular Scattering (1)
\begin{frame}
\frametitle{Semi--Classical Molecular Scattering of Polarized Light}
\end{frame}


%%%%% DFT SECTION
\section{DFT}

% Hohenberg-Kohn DFT
\begin{frame}
\frametitle{Density Functional Theory (DFT)}

The Hohenberg--Kohn theorems gurantee the following:
\begin{enumerate}
  \item The total energy of a fermionic system is a unique functional of the density, $F[\rho(\mathbf{r})]$.
  \item $F[\rho(\mathbf{r})]$ obtains its minimum value iff it is evaluated at the ground state density $\rho_G$.
\end{enumerate}

\textbf{Pros:}
\begin{itemize}
  \item Formally exact given $F$
  \item Trivially extends to relativistic treatments.
  \item No orbitals!
\end{itemize}

\textbf{Cons:}
\begin{itemize}
  \item $F$ is not known for non--trivial systems (jellium)
  \item Kinetic energy contributions in this form is exceptionally difficult.
\end{itemize}

\end{frame}

% Kohn-Sham DFT
\begin{frame}
\frametitle{Kohn--Sham Density Functional Theory (KS--DFT)}

Kohn and Sham introduced a method to obtain the kinetic energy contribution
within a mean field, Slater determinant picture
\begin{equation*}
\mathcal{H}^{KS} = \frac{1}{2}\Delta + \mathcal{V}(\mathbf{r}) + 
  \int \mathrm{d}^3\mathbf{r}'\text{ } \frac{\rho(\mathbf{r}')}{\vert \mathbf{r-r}' \vert} +
  \frac{\delta E^{xc}[\rho(\mathbf{r})]}{\delta \rho(\mathbf{r})}
\end{equation*}

\begin{equation*}
\mathcal{H}^{KS} \phi_i(\mathbf{r}) = \epsilon_i\phi_i(\mathbf{r})
\end{equation*}


\textbf{Pros:}
\begin{itemize}
  \item Again, formally exact given the exact $E^{xc}$
  \item Able to easily evaluate kinetic energy through the introduction
  of molecular orbitals.
  \item Still trivially extendable to relativistic theory
\end{itemize}

\textbf{Cons:}
\begin{itemize}
  \item $E^{xc}$ is still not known (we have to guess)
  \item Orbitals introduce complications over density--only methods
\end{itemize}
\end{frame}

% KSDFT Finite Basis
\begin{frame}
\frametitle{KS--DFT within a Finite Basis}

Casting into the mean--field picture allows for a simple expansion
of the KS--DFT equations in a finite basis $\{\chi(\mathbf{r})\}$
\begin{equation*}
F_{\mu\nu}^{KS} C_{\nu p} = S_{\mu\nu}C_{\nu p} \epsilon_p \qquad \quad
F_{\mu\nu}^{KS} = h_{\mu\nu} + J_{\mu\nu} + V^{xc}_{\mu\nu}
\end{equation*}

\begin{equation*}
X_{\mu\nu} = \int \mathrm{d}^3\mathbf{r}\text{ } \chi_\mu(\mathbf{r}) \mathcal{X} \chi_\nu(\mathbf{r})
\qquad \quad V^{xc}_{\mu\nu} = \int \mathrm{d}^3\mathbf{r}\text{ } 
  \frac{\delta E^{xc}[\rho(\mathbf{r})]}{\delta \rho(\mathbf{r})}
  \chi_\mu(\mathbf{r}) \chi_\nu(\mathbf{r})
\end{equation*}

Each molecular orbital is then expanded in the basis
\begin{equation*}
\phi_p(\mathbf{r}) = C_{\mu p} \chi_\mu(\mathbf{r})
\end{equation*}
\end{frame}

%%%%% TDDFT SECTION
\section{TD-DFT}

\begin{frame}
\frametitle{Real--Time Time--Dependent Density Functional Theory (RT-TD-DFT)}
\end{frame}

%%%%% CHEBYSHEV SECTION
\section{Chebyshev}

\begin{frame}
\frametitle{Chebyshev Expansion of the Quantum Propagator}
\end{frame}

%%%%% RESULTS SECTION
\section{Results}

\begin{frame}
\frametitle{Na D--Line Splitting}
\end{frame}




\end{document}

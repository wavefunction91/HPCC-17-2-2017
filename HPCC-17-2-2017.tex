\documentclass{beamer}
\usetheme[compress]{Singapore}
\setbeamerfont{footnote}{size=\tiny}


\usepackage{amsmath}
\usepackage{mathrsfs}
\usepackage{amssymb}
\usepackage{feynmp}
\usepackage{bm}


\newcommand{\bpar}[1]{\left( #1 \right)}                  % big parentheses
\newcommand{\bra}[1]{\left \langle #1 \right \vert}
\newcommand{\ket}[1]{\left \vert #1 \right \rangle}
\newcommand{\comm}[2]{\left[ #1 , #2\right]}
\newcommand{\qop}[1]{\mathscr{#1}}
\newcommand{\mat}[1]{\boldsymbol{#1}}
\newcommand{\dbar}[2]{\left\langle \left. #1 \right\vert \left\vert #2 \right.\right\rangle}

\title[]{Studying Semi--Classical Molecular Light--Matter Interaction through
Time--Dependent Density Functional Theory (TD-DFT)}
\author[Li Research Group, University of Washington]{
\\[1\baselineskip]
David Williams--Young \\
Department of Chemistry \\
University of Washington \\
Li Research Group
}

\date{\today}

\begin{document}

% Title Page
\begin{frame}
\titlepage
\end{frame}


%%%%% OUTLINE SECTION
\section{Outline}


% Outline
\begin{frame}
\frametitle{Outline}

\begin{itemize}
  \item Semi--Classical Molecular Scattering of Polarized Light
  \item Density Functional Theory (DFT)
  \item Real--Time Time--Dependent Density Functional Theory (RT-TD-DFT)
  \item Chebyshev Expansion of the Quantum Propagator
  \item Some Fun Results
\end{itemize}
\end{frame}

%%%%% THEORY SECTION
\section{Theory}

% Molecular Scattering (1)
\begin{frame}
\frametitle{Semi--Classical Molecular Scattering of Polarized Light}
\end{frame}


%%%%% DFT SECTION
\section{DFT}

\begin{frame}
\frametitle{Density Functional Theory (DFT)}
\end{frame}

%%%%% TDDFT SECTION
\section{TD-DFT}

\begin{frame}
\frametitle{Real--Time Time--Dependent Density Functional Theory (RT-TD-DFT)}
\end{frame}

%%%%% CHEBYSHEV SECTION
\section{Chebyshev}

\begin{frame}
\frametitle{Chebyshev Expansion of the Quantum Propagator}
\end{frame}

%%%%% RESULTS SECTION
\section{Results}

\begin{frame}
\frametitle{Na D--Line Splitting}
\end{frame}




\end{document}
